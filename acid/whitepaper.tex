\documentclass{article}
\usepackage{multicol}
% Packages
\usepackage{lipsum} % For generating dummy text
\usepackage[top=1in, bottom=1in, left=1in, right=1in]{geometry}
\usepackage{hyperref}
% Title and author
\title{Autonomous competence identification protocol}
\author{Tim Pechersky}


\begin{document}

\maketitle



\begin{abstract}
    The proposed protocol offers a novel approach where ranking system is built based on peer to peer network topology based on already established existing precedent.
    The core idea is simple and can be explained for non technical person as a simple game that can be played with friends, family or colleagues, while it is also scalable to machine to machine interactions.  The existing precedent of such game is studied and shows some optimistic participation rate numbers.
    On more technical level the protocol aims to address the challenge of finding consensus in social interactions and decision-making processes. It focuses on defining interoperable and liquid voting weights for participants in a non-computational system, specifically in the context of decentralized autonomous organizations (DAOs).
\end{abstract}
\begin{multicols}{2}
\section{Introduction}
The proposed protocol offers a novel approach to competence identification and decision-making based on peer to peer network topology based on already established existing precedent.
The core idea is simple and can be explained for non technical person as a simple game that can be played with friends, family or colleagues, while it is also scalable to machine to machine interactions.  The existing precedent of such game is studied and shows some optimistic participation rate numbers.
On more technical level the protocol aims to address the challenge of finding consensus in social interactions and decision-making processes. It focuses on defining interoperable and liquid voting weights for participants in a non-computational system, specifically in the context of decentralized autonomous organizations (DAOs).    The proposed protocol offers a novel approach to competence identification and decision-making based on peer to peer network topology based on already established existing precedent.
The core idea is simple and can be explained for non technical person as a simple game that can be played with friends, family or colleagues, while it is also scalable to machine to machine interactions.  The existing precedent of such game is studied and shows some optimistic participation rate numbers.
On more technical level the protocol aims to address the challenge of finding consensus in social interactions and decision-making processes. It focuses on defining interoperable and liquid voting weights for participants in a non-computational system, specifically in the context of decentralized autonomous organizations (DAOs).    The proposed protocol offers a novel approach to competence identification and decision-making based on peer to peer network topology based on already established existing precedent.
The core idea is simple and can be explained for non technical person as a simple game that can be played with friends, family or colleagues, while it is also scalable to machine to machine interactions.  The existing precedent of such game is studied and shows some optimistic participation rate numbers.
On more technical level the protocol aims to address the challenge of finding consensus in social interactions and decision-making processes. It focuses on defining interoperable and liquid voting weights for participants in a non-computational system, specifically in the context of decentralized autonomous organizations (DAOs).    The proposed protocol offers a novel approach to competence identification and decision-making based on peer to peer network topology based on already established existing precedent.
The core idea is simple and can be explained for non technical person as a simple game that can be played with friends, family or colleagues, while it is also scalable to machine to machine interactions.  The existing precedent of such game is studied and shows some optimistic participation rate numbers.
On more technical level the protocol aims to address the challenge of finding consensus in social interactions and decision-making processes. It focuses on defining interoperable and liquid voting weights for participants in a non-computational system, specifically in the context of decentralized autonomous organizations (DAOs).\cite{lamport1994}

\section{Background}
% Write your methodology here



\section{Protocol Description}
% Write your conclusion here
\section{Implementation}

% Write your results here
\section{Case Study}

\section{Conclusion}

\end{multicols}{2}
\bibliographystyle{ieeetr}
\bibliography{whitepaper.bib}


\end{document}