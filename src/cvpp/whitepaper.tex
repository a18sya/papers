\documentclass{article}
\usepackage{CJKutf8}
\usepackage{multicol}
% Packages
\usepackage{lipsum} % For generating dummy text
\usepackage[top=1in, bottom=1in, left=1in, right=1in]{geometry}
\usepackage{hyperref}
\usepackage{pgfplots}
\usepackage{caption}
\captionsetup{font=small}
\usepackage{standalone}
\usepackage{listings}
\usepackage{xcolor} % For setting colors
\usepackage{amssymb}
\usepackage{amsmath}
\usepackage{algorithm}
\usepackage{algpseudocode}
\usepackage{afterpage}
\usepackage{placeins}
\usepackage{tikz}
\usepackage{enumitem}
\lstset{
    language=[LaTeX]TeX,
    breaklines=true,
    basicstyle=\ttfamily\small,
    keywordstyle=\color{blue},
    commentstyle=\color{green},
    backgroundcolor=\color{gray!10},
    frame=single,
    showspaces=false,
    showstringspaces=false,
}
\pgfplotsset{compat=1.17} % Use this to ensure compatibility with newer features
\setlength{\parskip}{6pt}
% Title and author
\title{Continuous Voting-Proposing Protocol for Ordering Group Intents}

\author{Tims Pecerskis, Aivars Smirnovs, Alesya Soboleva}


\begin{document}
\begin{CJK}{UTF8}{gbsn}

    % \twocolumn
    \maketitle

    \begin{abstract}
        We present the Continuous Voting Proposing Protocol (CVPP), a novel framework for ordering group intents and facilitating collaborative decision-making. The protocol combines continuous voting with proposal mechanisms, enabling consensus while maintaining individual autonomy. Through empirical evaluation in music playlist curation, we demonstrate its effectiveness with high user satisfaction rates (88\% overall experience, 92\% playlist quality) and its ability to surface diverse, high-quality proposals. Applications extend to software development, decentralized autonomous organizations (DAOs), and delegate selection, offering a flexible foundation for democratic systems while mitigating coordination challenges like low participation and bias. Our results show that CVPP can balance individual preferences with group consensus, providing a gamified yet practical framework for collective decision-making and AI agent training.
    \end{abstract}
    \begin{multicols}{2}

        \section{Introduction}

        This paper presents a continuous voting-proposing protocol inspired by an online text-based tabletop game played by Lithuanian artists abroad to maintain connections with friends and family across time zones and countries based on shared interest topics.

        In this paper, we extrapolate the protocol from this real-life case, analyze its properties, and review the statistics and engagement metrics of the existing use case, enhanced with our prototype study.

        \section{Background}

        The quest for consensus is a cornerstone of collective decision-making with deep historical roots. From the ancient Chinese concept of \textit{zhongyong} {\CJKfamily{bsmi}中庸}, advocating moderation and balance in governance, to the Roman Republic's emphasis on \textit{senatus consulta} (senate decrees) reached through deliberation and compromise. The study of these historical precedents informs modern approaches to consensus \cite{Andersen2019} \cite{Frederic2014}.

        \paragraph{Decision making in research} has a backlog of unresolved problems, including agenda manipulation that shows democratic voting outcomes in multiple iterations have non-transitivity, which may lead to any desired outcome given limited items\cite{McKelvey1976}. This problem, described back in the 80s, is now prominent due to advancements in AI and market studies, meaning better ability to predict preferences. Participation rates (turnout) are declining in major countries over the past decades \cite{Lawrence23}\cite{Filip24}, and recent research on Decentralized Autonomous Organizations (DAOs) confirms that\cite{Rainer2023}\cite{Marcella2016}\cite{Xuan2024}.
        \paragraph{Meritocratic models} have been hampered by a fundamental issue: the reliable and autonomous identification of competence, where leadership is based on ability. Researchers have argued\cite{Arrow2000} that the inability to objectively assess merit creates vulnerabilities to manipulation and power struggles, limiting the practical application of such systems.
        \paragraph{Proposed Protocol Origins}\label{sec:protocol_origins}
        This research stems from a game played by a small Lithuanian community in online chat since 2017. This tabletop game, initially designed for leisure and staying in touch with friends, has a goal to share music to form collaborative playlists ordered by the highest-ranking proposal. It has exhibited properties desirable in governance protocols.
        Participants’ goal is to create a 100-song YouTube playlist ordered by the highest scoring song and to create a community-governed charter\cite{DariusYoutube}.
        This protocol is limited to a small group of participants (5-10) due to communication complexity, but it shows great properties with a moderate count.
        It shows high participation and retention rates, suggesting potential for adaptation as a governance protocol (elaborated in case study section). The primary limitation is participants' inclination towards specific interests (music) or groups.
        \paragraph{Empirical data.} We analyzed a group's activity and records in a music-sharing game \cite{DariusYoutube}. Participants share YouTube music clips, forming playlists. From 2017 to 2023, 17 games were played, averaging 13 turns, 8 participants, and 3 months duration. About 1700 proposals and votes were recorded, with a 100-song playlist limit per game. Over the years, 34 participants joined, 22 completed at least one tournament, and 15 completed at least two. Of the original 6 players, 3 remain active, with 8 users in the latest game. Non-voting instances were negligible, below 5\% of all votes. \\
        Participation rates were high, despite lacking explicit incentives and occasional game master duties:
        \begin{itemize}[nosep]
            \item \textbf{Participation rate} of 95\% for active users
            \item \textbf{User retention} for three months of 65\%
            \item \textbf{User retention} for two tournaments and six months of 44\%
            \item \textbf{User retention} after 72 months of 9\%
            \item Average of \textbf{8 proposals and votes per user} per year
        \end{itemize}

        However, the low absolute numbers are due to the complex manual score calculation process and a closed community that never aimed for big growth. This suggests higher numbers on a well-designed automated protocol.

        We observed that high-score ordered playlists are of high quality and represent participants' preferences. The tournament winner can propose a song liked by many, including aspects that participants will vote for new songs (original ideas).

        \section{Protocol Description}
        The protocol involves a rotating moderator, or "game master" role. This is a participant who at the given turn won't vote or propose and will facilitate the protocol manually. The game is split into multiple rounds with a typical round duration dependent on use case and proposal assessment time.

        If the player count exceeds the proposal goal, the playlist size may increase to ensure every participant shares equal load in facilitating the moderator role.
        \paragraph{Proposing stage.}Each round, participants must privately submit their proposals to the moderator. When all are aggregated or a pre-arranged timeout is reached, the moderator compiles and publishes the actual proposal list, without disclosing proposers’ identities.
        \paragraph{Voting stage.}Participants privately submit their vote to the moderator on three proposals they like, giving them 3,2 and 1 points accordingly. Skipping a vote gives everyone but the participant 3 points, offsetting inactive scores below others even if they sent a proposal. Voting for own proposal is disallowed.
        If the proposal wasn’t sent in time, the participant is considered inactive and the moderator might not wait for their votes. However, if it’s accepted.
        \paragraph{Reveal stage.}After the timeout or all actions are made, the moderator reveals the results, including proposer identities and aggregated proposal scores. This public announcement allows participants to discuss the results and gain insights into other members’ preferences.
        \paragraph{Final stage}Participants iterate through the three stages until the desired count of proposals is reached. The participant with the highest aggregate score of proposals wins, and the high-score ordered chart of songs is finalized.
        \paragraph{Boundary cases.} There are few boundary cases, and they are simple: (i) submission of duplicate proposals in one or various cycles: these can be merged, creating a "conviction" voting system, where the same proposal can be lifted higher over multiple cycles (ii) if there are multiple highest scoring participants, split the winning title or continue additional cycles until a distinct leader emerges.

        \section{Prototype study}
        We wrote a simple implementation of a protocol that allows easy process automation. Then, we conducted two tournaments, each with 9 participants.
        Participants included original concept participants and random people of different age groups who did not know each other.

        The tournament rules and requirements were set close to original: participants had to send YouTube URLs for music compositions no longer than 10 minutes and listen to all to vote.

        A backend server ensured privacy constraints. A quadratic voting system allowed users to spend 14 vote credits in integer form. The only notable difference was that for design simplification, the total playlist length was replaced with a fixed round count.

        Participants had no incentives and were driven solely by their interest in music. During these two tournaments, some participation metrics were recorded:

        \begin{itemize}[nosep]
            \item \textbf{Participation rate} of 84\% for active users
            \item \textbf{User retention} of 66.7\% after first tournament
            \item Average of \textbf{3.23 proposals and 3.09 votes per user} per month
        \end{itemize}

        Participants were asked to fill a feedback form with questions rated from 0 to 5 based on experience.
        \begin{itemize}
            \setlength\itemsep{2px}
            \item \textbf{Overall experience rating:} 88\%.
            \item \textbf{Resulting playlists rating:} 92\%.
            \item \textbf{\textit{"Did playing this game give you new insights into other players?"}}: rated 80\%.
            \item \textbf{\textit{"Would you play such a game with your friends or family?"}}: rated 84\%.
        \end{itemize}

        \paragraph{Turnover participants}. Most turnout respondents referred to lack of (i) interest in the protocol implementation subject or (ii) other proposals, but showed interest in participating with their own group of friends.

        Participants proposed various tournament topics and expressed interest. \\When asked, \textit{"If you missed any turn to vote or propose, what were the main causes"}, all respondents said they enjoyed the process but lacked time to listen to all compositions.

        \section{Potential Applications}
        The protocol accommodates various voting setups. It can handle traditional voting cases of few agenda items and many voters by allowing only X of N to submit proposals.
        It allows a "conviction" voting system where proposal weights increase over multiple rounds if voted for the same proposal. It also accommodates "conviction election" where a leader can continuously propose something chosen by the group may be seen as a competent delegate.

        It allows extending "conviction" with a quadratic voting system. The original implementation implements a spread of points across the top 3 proposals.\cite{Buterin20}\cite{Benhaim2024}.

        Implementing non-private voting is straightforward as agents (or moderators) may announce their proposal either freely or by protocol modification.

        Implementing this protocol will create a superset of all known voting systems and advance traditional governance.

        This is useful in the following applications:

        \paragraph*{Academic Research and Peer Review:} The traditional peer-review process often suffers from biases, delays, and lack of transparency. The autonomous competence identification protocol can address these issues by creating a decentralized system. Researchers can submit their work, review others' submissions, and receive ratings based on the quality and thoroughness of their reviews. This approach leads to faster, more objective, and transparent peer review, improving the quality and credibility of academic research.

        \paragraph*{Online Education and Skill Assessment:} The protocol can be applied to online education platforms to assess and certify learners' skills. They can demonstrate their knowledge and proficiency by participating in competence tournaments or challenges. The ranking system provides a verifiable and transparent way to assess skills, enabling learners to showcase their expertise to potential employers or educational institutions.

        \paragraph{Agent empathy training} protocol establishes a framework for rating participants and ideas, decoupling evaluation from personal biases. Its continuous feedback loop, utilizing previous round results, fosters deeper alignment among participants and cultivates empathy as they refine their understanding of collective preferences.

        This property holds significant value for Cyber-Physical-Social Systems (CPSS)\cite{Fei2016}. Beyond interpersonal empathy, the protocol generates feedback and historical data to create robust, personalized machine learning frameworks by benchmarking AI agents in these tournaments.

        \paragraph{Decentralized charts} The first use case of the Protocol is to create a community-managed intents list that is a useful asset.

        \paragraph{Just-in-time decision management.} This protocol facilitates streamlined decision-making by obtaining the top-rated proposer and proposal just-in-time through the protocol design.

        \paragraph*{Collaborative Writing.} Developers submit pull requests, vote on solutions, and iteratively refine the codebase and documentation by treating each round as a work cycle.

        The fixed cycle length ensures a strict deadline, promoting efficiency and focus. Assigning extra weight multipliers to later rounds acknowledges the cumulative improvement of code over time, enabling automatic merging of best solutions. It encourages participants to contribute new code and solutions rather than just reviewing, fostering a proactive and collaborative development environment.

        \paragraph{Early conflict prediction} By observing participant voting in a setup abstracting ideas from personalities, agent opinion misalignment can be detected early. This serves as a conflict prediction flag.
        \paragraph*{Decentralized Autonomous Organizations}: DAOs can facilitate subject-specific discussions, countering low participation rates and turnover.\cite{Rainer2023}\cite{Marcella2016}\cite{Xuan2024}
        \paragraph{Delegate finding} Groups can facilitate a protocol to find a fair delegate with a framework that avoids halo effect problems\cite{Verhulst2010} and provides unbiased recognition to the successful member.

    \end{multicols}
    \section{Conclusion}

    This paper introduced a novel protocol for social coordination. This protocol is a superset that incorporates existing democratic voting systems while providing freedom and flexibility to create new frameworks.
    It shows promising metrics in early evaluations and allows creating a simple, gamified experience for facilitating useful discussions or AI agent training.
    We've analyzed potential use cases for corporate governance, decentralized organizations, non-profit foundations, small households, and emerging communities.
    This protocol does not focus on establishing a social graph between participants, but rather on reinforcing it, as the ability to reinforce empathy and identify conflicts early allows us to.
    Further research is needed to see how this protocol could serve as a foundation for social, intents-based BFT protocols.


    \clearpage

    % \appendix
    % \section{Appendix A: Listings}
    % 
\begin{algorithm}
\caption{Basic Autonomous competence identification}
\begin{algorithmic}[1] % Enables line numbering
\State \textbf{Prerequisites:} Minimal group size $N_{min}$, Time constant $t_{c}$, Participation cost $X_g$, Competence Asset $C_i$, Protocol Treasury $T$, Minimal turn count ${M, M > 2}$
\State \textbf{Process:}
\Procedure{Preparation}{}
    \State Instance is created at timestamp $t_0=t$, with creators personal join requirements $X_a$
    \State Participants join the group and lock in a commitment: $N_{min} \times (X_g+X_a) \rightarrow T$
    \State Joining period ends at $t_1  \ge t_c + t_0$
\EndProcedure
\Procedure{FirstTurn}{}
    \While{$t_1 + t_c \ge t$}
        \If{new Proposal}
        \State Record ${P_i}$
        \State Record $i$
        \EndIf
    \EndWhile
    \If{Group is full or joining timeout}
        \State Members exchange encrypted proposals (MPC).
    \EndIf
\EndProcedure
\Procedure{DecryptProposals}{}
    \If{Timeout or all proposals received}
        \State Members decrypt proposals using threshold encryption.
    \EndIf
\EndProcedure
\Procedure{RegularTurn}{}
    \State Submit new encrypted proposals.
    \State Submit encrypted votes and zk-proof of valid vote.
\EndProcedure
\Procedure{CalculateRoundScores}{}
    \If{Round ends or all votes \& proposals submitted}
        \State Use TSS cryptography to decrypt everything.
        \State Calculate round scores.
    \EndIf
\EndProcedure
\Procedure{RepeatUntilXProposals}{}
    \State Repeat steps 3-5 until a total of X proposals.
\EndProcedure
\Procedure{FinalizeRankings}{}
    \State Count total votes and sort proposals.
\EndProcedure
\end{algorithmic}
\end{algorithm}


    % \clearpage
    \bibliographystyle{ieeetr}
    \bibliography{../references.bib}

    \clearpage\end{CJK}
\end{document}
